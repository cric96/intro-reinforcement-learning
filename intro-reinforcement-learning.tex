%%%%%%%%%%%%%%%%%%%%%%%%%%%%%%%%%%%%%%%%%%%%%%%%%%%%%%%%%%%%%%%%%%%%%%%%%%%%%%%%
% AMS Beamer series / Bologna FC / Template
% Andrea Omicini
% Alma Mater Studiorum - Università di Bologna
% mailto:andrea.omicini@unibo.it
%%%%%%%%%%%%%%%%%%%%%%%%%%%%%%%%%%%%%%%%%%%%%%%%%%%%%%%%%%%%%%%%%%%%%%%%%%%%%%%%
%\documentclass[handout]{beamer}\mode<handout>{\usetheme{default}}
%
\documentclass[presentation, 9pt]{beamer}\mode<presentation>{\usetheme{AMSBolognaFC}}
%\documentclass[handout]{beamer}\mode<handout>{\usetheme{AMSBolognaFC}}
%%%%%%%%%%%%%%%%%%%%%%%%%%%%%%%%%%%%%%%%%%%%%%%%%%%%%%%%%%%%%%%%%%%%%%%%%%%%%%%%
\usepackage[T1]{fontenc}
\usepackage{wasysym}
\usepackage{amsmath,blkarray}
\usepackage[minted,most]{tcolorbox}
\usepackage{centernot}
\usepackage{fontawesome}
\usepackage{fancyvrb}
\usepackage{minted}
\usepackage{hyperref}
\usepackage{multicol}
\setminted[scala]{fontsize=\scriptsize,baselinestretch=1,obeytabs=true, tabsize=2}
\usepackage[ddmmyyyy]{datetime}
\setminted{fontsize=\footnotesize}
\renewcommand{\dateseparator}{}
%\renewcommand{\thefootnote}{\fnsymbol{footnote}}
\newcommand{\version}{1}
\usepackage[
	backend=biber,
	citestyle=authoryear-icomp,
	maxcitenames=1,
	bibstyle=numeric]{biblatex}

	\makeatletter

\addbibresource{biblio.bib}
%%%%%%%%%%%%%%%%%%%%%%%%%%%%%%%%%%%%%%%%%%%%%%%%%%%%%%%%%%%%%%%%%%%%%%%%%%%%%%%%
\title[Reinforcement Learning]
{Reinforcement Learning}
\subtitle{An introduction}
%
%
\author[\sspeaker{Aguzzi}]
{\speaker{Gianluca Aguzzi} \href{mailto:gianluca.aguzzi@unibo.it}{gianluca.aguzzi@unibo.it}}
%
\institute[DISI, Univ.\ Bologna]
{Dipartimento di Informatica -- Scienza e Ingegneria (DISI)\\
\textsc{Alma Mater Studiorum} -- Universit{\`a} di Bologna \\[0.5cm]
\textbf{Talk @} \bold{Advanced School in Artificial Intelligence (ASAI)}}
%
\renewcommand{\dateseparator}{/}
\date[\today]{\today}
%
\AtBeginSection[]
{
  \begin{frame}
  \frametitle{Contents}
  \tableofcontents[currentsubsection, 
	sectionstyle=show/shaded, 
	subsectionstyle=show/shaded]
  \end{frame}
}
\AtBeginSubsection[]
{
  \begin{frame}
  \frametitle{Contents}
  \tableofcontents[currentsubsection, 
	sectionstyle=show/shaded, 
	subsectionstyle=show/shaded]
  \end{frame}
}
%%%%%%%%%%%%%%%%%%%%%%%%%%%%%%%%%%%%%%%%%%%%%%%%%%%%%%%%%%%%%%%%%%%%%%%%%%%%%%%%
\begin{document}
%%%%%%%%%%%%%%%%%%%%%%%%%%%%%%%%%%%%%%%%%%%%%%%%%%%%%%%%%%%%%%%%%%%%%%%%%%%%%%%%

%/////////
\frame{\titlepage}
%/////////
\begin{frame}{Hello World!}
	\begin{columns}
		\begin{column}{0.5\textwidth}
		\centering
		\fbox{\includegraphics[width=0.5\linewidth]{img/me.jpeg}}
		\\
		\vspace{0.2cm}
		\href{https://github.com/cric96}{\faGithub} \,
		\href{https://stackoverflow.com/users/10295847/gianluca-aguzzi}{\faStackOverflow} \,
		\href{https://www.linkedin.com/in/gianluca-aguzzi-265998170/}{\faLinkedin} \,
		\href{https://www.unibo.it/sitoweb/gianluca.aguzzi}{\faGlobe} \,
		\end{column}
		\begin{column}{0.5\textwidth}
			\begin{itemize}
				\item PhD student in Computer Science and Engineering
				\item Research interests:
				\begin{itemize}
					\item Multi-agent systems
					\item Distributed Collective Intellingence
					\item Deep Reinforcement Learning
					\item Multi-agent Reinforcement Learning
					\item Distributed Macro-programming
				\end{itemize}
				%\item Lead developer of \href{https://scafi.github.io/}{ScaFi}
				%\item Scala Lover \& Functional Programming enthusiast
			\end{itemize}
		\end{column}
	\end{columns}
\end{frame}
\begin{frame}{Resources}
\begin{itemize}
	\item \emph{An Introduction to Reinforcement Learning}, Sutton and Barto,1998
	\begin{itemize}
		\item Available online at \url{http://incompleteideas.net/book/the-book-2nd.html}
	\end{itemize}
	\item \emph{Foundations of Deep Reinforcement Learning: Theory and Practice in Python}, Laura Graesser and Wah Loon Keng, 2020
	\item \emph{Deep Mind Lectures}:
	\begin{itemize}
		\item \textbf{Introduction to Reinforcement Learning with David Silver}: \url{https://www.deepmind.com/learning-resources/introduction-to-reinforcement-learning-with-david-silver}
		\item \textbf{Reinforcement Learning Lecture Series}: \url{https://www.deepmind.com/learning-resources/reinforcement-learning-lecture-series-2021}
	\end{itemize}
\end{itemize}
\end{frame}
%===============================================================================
\section{Introduction}
%===============================================================================
\begin{frame}[plain,c]
	%\frametitle{A first slide}
	\begin{center}
	\Huge What is Reinforcement Learning?
	\end{center}
\end{frame}

\begin{frame}[plain,c]
	%\frametitle{A first slide}
	\begin{center}
	\Huge What is Intelligence?
	\\
	\huge \emph{To able to \textbf{learn} to make \textbf{decisions} to achive \textbf{goals}}
	\end{center}
\end{frame}
\begin{frame}{What is Reinforcement Learning?}
\begin{itemize}
	\item Animals learn by interacting with our environment
	\begin{itemize}
		\item Babies learn how to communicate by interacting with parents
		\item Dogs learn how to behave by following the owner's orders
	\end{itemize}
	\item 
\end{itemize}
\end{frame}

%===============================================================================
\section*{}
%===============================================================================

%/////////
\frame{\titlepage}
%/////////

%===============================================================================
\section*{\refname}
%===============================================================================

%%%%
\setbeamertemplate{page number in head/foot}{}
%/////////
\begin{frame}[c,noframenumbering, allowframebreaks]{\refname}
%\begin{frame}[t,allowframebreaks,noframenumbering]{\refname}
	\tiny
	\nocite{*}
	\printbibliography
\end{frame}
%/////////

%%%%%%%%%%%%%%%%%%%%%%%%%%%%%%%%%%%%%%%%%%%%%%%%%%%%%%%%%%%%%%%%%%%%%%%%%%%%%%%%
\end{document}
%%%%%%%%%%%%%%%%%%%%%%%%%%%%%%%%%%%%%%%%%%%%%%%%%%%%%%%%%%%%%%%%%%%%%%%%%%%%%%%%
